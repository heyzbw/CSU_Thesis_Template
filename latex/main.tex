\documentclass[a4paper,12pt,UTF8,oneside,hyperref]{ctexart}
%,twoside
\usepackage{times}%全文英文字体使用Times New Roman;
\linespread{1.5}%全文1.5倍行距;
%========================电子文档与超链接======================
%\usepackage[linkcolor=blue,citecolor=blue,backref=page,hyperfootnotes=false]{hyperref}
\usepackage[backref=page,hyperfootnotes=false]{hyperref}
\usepackage[a4paper,left=3.0cm,right=2.0cm,top=3.51cm,bottom=4.25cm]{geometry}
\hypersetup{
% colorlinks = true,
% linkcolor=blue,
 bookmarks = true,
 bookmarksopen = true,
 pdfborder=0 0 0,
 hyperfootnotes=false,
% pdfpagemode = FullScreen,
 pdfstartview = Fit,
 pdftitle = {本科毕业设计(论文)},
 pdfauthor = {heyzbw}
}
\pdfbookmark[1]{Contents}{contents}
%===============================================================

%=========================设=置=页=眉===========================
\usepackage{fancyhdr}% 为了保证正确长度的页眉下划线,fancyhdr宏包要在geometry宏包后使用;
\pagestyle{fancy}
\fancyhf{}
\lhead{\includegraphics[scale=0.13]{./pictures/CSU_Logo.jpg}}
\usepackage{lastpage}
\rhead{\small\heiti 本科毕业设计(论文模板)}
%===============================================================

%==========================章节格式=============================
\usepackage{titlesec}
\usepackage{indentfirst}
%\usepackage{zhnumber} % 章节格式为中文
%\renewcommand\thesection{\zhnum{section}}
%\renewcommand \thesubsection {\arabic{section}}
\setlength{\parindent}{2em}
\titleformat{\section}[hang]{\heiti\Large\center}{第\,\thesection\,章 \quad}{0pt}{}{}
%\titleformat{\subsection}[hang]{\heiti\normalsize\bfseries\raggedright}{\thesubsection\quad}{0pt}{}{}
\titleformat{\subsection}[hang]{\heiti\normalsize\raggedright}{\thesubsection\quad}{0pt}{}{}
\titleformat{\subsubsection}[hang]{\normalsize\kaishu\raggedright}{\thesubsubsection\quad}{0pt}{}{}
\titlespacing*{\section}{0pt}{0pt}{19.2pt}[0pt]
\titlespacing*{\subsection}{0pt}{0pt}{0pt}[0pt]
\titlespacing*{\subsubsection}{0pt}{0pt}{0pt}[0pt]
%===============================================================

%============================目录===============================
\usepackage{titletoc}
\titlecontents{section}[0mm]{}{}{}{}[]
\dottedcontents{section}[0.0em]{\normalsize}{1.0em}{4pt}
\dottedcontents{subsection}[3.0em]{\normalsize}{2.0em}{4pt}
\dottedcontents{subsubsection}[6.0em]{\normalsize}{3.0em}{4pt}
%===============================================================

%==========================脚注格式=============================
\usepackage{pifont}%更好看的一个脚注格式
\renewcommand\thefootnote{\ding{\numexpr171+\value{footnote}}}% 更好看的一个脚注格式
\usepackage[perpage]{footmisc}%脚注每页清零编号
%===============================================================

%==========================插入图片=============================
\usepackage{graphicx}
\usepackage{subfigure}%两图并排
\usepackage{picinpar}%文字绕排
\usepackage{caption}
\captionsetup{font=small,labelfont=bf,labelsep=quad}
\captionsetup{skip=0pt}%标题与浮动环境内容的垂直间距无额外间距(设置为0pt);
%===============================================================

%==========================插入表格=============================
\usepackage{booktabs,multirow}%三线表
\usepackage{longtable}%长表格
\usepackage{bigstrut}
% \usepackage{subcaption}
\usepackage{diagbox}
\usepackage{makecell}
\usepackage{tabularx}
\newcolumntype{Y}{>{\centering\arraybackslash}X}
%==========================解决三线表竖线过短的问题=============================
\renewcommand{\cmidrulesep}{0mm} %定义两条相邻\cmidrule之间的间隔
\setlength{\aboverulesep}{0mm} %在线条[不包括\toprule]上面增加一段垂直距离,此处为0mm
\setlength{\belowrulesep}{0mm} %在线条[不包括\bottomrule]下面增加一条垂直距离,此处为0mm
\setlength{\abovetopsep}{0cm}  %在线条\toprule上面,即表格与上面的文字之间的距离。
\setlength{\belowbottomsep}{0cm}%在线条\bottomrule下面,即表格与下面的文字之间的距离。
%===============================================================

%==========================数学公式=============================
\usepackage{amsmath}%公式
\usepackage{amssymb}
\usepackage{bm}%粗体
\usepackage{upgreek}%直立希腊字母
\usepackage{mhchem}%化学公式 \ce{SiO2} 或 \ce{H2O}
\DeclareMathOperator\dif{d\!}%微分算子\frac{\dif y}{dif x}
%\usepackage{xfrac}%行内小分式
\newenvironment{sequation}{\begin{equation}\small}{\end{equation}}
\newenvironment{tequation}{\begin{equation}\tiny}{\end{equation}}
%===============================================================

%==========================程序代码=============================
\usepackage{listings}
%===============================================================

%==========================使用彩色=============================
%\usepackage[usenames,dvipsnames]{color}
\usepackage{color}
\usepackage[table]{xcolor}
\usepackage{colortbl}
\definecolor{mygray}{gray}{.9}
\definecolor{mypink}{rgb}{.99,.91,.95}
\definecolor{mycyan}{cmyk}{.3,0,0,0}
%================================================================

%=============================其它===============================
\usepackage{siunitx}%摄氏度 \SI{100}{\degreeCelsius}
\makeatletter%罗马数字
\newcommand{\rmnum}[1]{\romannumeral #1}% 罗马数字
\newcommand{\Rmnum}[1]{\expandafter\@slowromancap\romannumeral #1@}%罗马数字
\makeatother%罗马数字 \Rmnum{1} 或 \Rmnum{2} 或 \rmnum{3}
\usepackage{xfrac}%行内小分式
%==================================================================

%=============================文献格式===============================
\usepackage{gbt7714}%文献格式使用GB/T7714-2015标准;
\newcommand{\upcite}[1]{\textsuperscript{\textsuperscript{\cite{#1}}}}
\usepackage{natbib}
\renewcommand\bibfont{\small\kaishu}%文献字体改用5号字体;
\setlength\bibsep{0pt}%取消不同文献条目之间的距离;
\usepackage{url}
%==================================================================

%=========================一个更好的下划线设置=======================
\usepackage{ulem}

\usepackage{enumitem}

\usepackage{hhline}% 双线表

\newcommand\dlmu[2][4cm]{\hskip1pt\underline{\hb@xt@ #1{\hss#2\hss}}\hskip3pt}

%=========================公式按章编号===============================
\makeatletter
\@addtoreset{equation}{section}
\makeatother
\renewcommand{\theequation}{\arabic{section}-\arabic{equation}}

\makeatletter
\@addtoreset{figure}{section}
\makeatother
\renewcommand{\thefigure}{\thesection\,-\,\arabic{figure}}

\makeatletter
\@addtoreset{table}{section}
\makeatother
\renewcommand{\thetable}{\thesection\,-\,\arabic{table}}
%==================================================================

\usepackage{ifpdf}

\newtheorem{thm}{定理}

%==============积分符号修改,改原倾斜的积分符号为直立体==================
\usepackage{amsmath,amssymb}
\DeclareSymbolFont{EulerExtension}{U}{euex}{m}{n}
\DeclareMathSymbol{\euintop}{\mathop} {EulerExtension}{"52}
\DeclareMathSymbol{\euointop}{\mathop} {EulerExtension}{"48}
\let\intop\euintop
\let\ointop\euointop
%==================================================================

\usepackage{doc}%BibTeX显示;
\usepackage{pythonhighlight}
\begin{document}
\begin{titlepage}
\phantom{\LARGE 中南大学}
\begin{figure}[htbp]
  \centering
  \includegraphics[scale=0.58]{./pictures/CSU_Logo.jpg}\\
\end{figure}

\phantom{中南大学}

\begin{center}
{\fontsize{45pt}{14.4pt}\heiti 本科毕业设计(论文)}
\end{center}

\begin{center}
 {\zihao{1}\rmfamily GRADUATION DESIGN(THESIS)}
\end{center}

\phantom{中南大学}

\phantom{中南大学}

\begin{table}[htbp]
\centering
\LARGE\kaishu
\begin{tabular}{cc}
  {\heiti 题\qquad 目:} & \makecell{本科毕业设计论文模板} \\
  \cline{2-2}
  {\heiti 学生姓名:} &  heyzbw \\
  \cline{2-2}
  {\heiti 指导老师:} & heyzbw \\
  \cline{2-2}
  {\heiti 学\qquad 院:} & 自动化学院 \\
  \cline{2-2}
  {\heiti 专业班级:} & 智能2002班 \\
  \cline{2-2}
\end{tabular}
\end{table}
\vspace{28pt}
\begin{center}
\zihao{2}{\heiti 本科生院制}\\

\zihao{-2}{\heiti 2022年10月}

\end{center}
\end{titlepage}


\clearpage
\phantom{s}
%\thispagestyle{empty}
%\newpage

\setcounter{page}{1}
\pagenumbering{Roman}
\cfoot{\small\thepage}
\vspace*{-21.6pt}
\begin{center}
\zihao{-2}\heiti
本科毕业设计(论文模板)
\end{center}
\par
\begin{center}
	\zihao{3}\heiti
	\phantomsection
	\addcontentsline{toc}{section}{摘要}
	摘要
\end{center}

这里是摘要,请好好写。本文的参考文献用于举例参考文献引用方法\cite{IOTB2}。本文的附录用于举例附录引用方法如“见附录\ref{matlab}和附录\ref{python}”。
\par

\bigskip
\noindent\textbf{关键字:}关键字1\quad 关键字2\quad 关键字3

\newpage
\vspace*{-21.6pt}
\begin{center}
\zihao{-2}\bfseries
\rmfamily Template for the Thesis of Central South University
\end{center}
\par
\phantomsection
\addcontentsline{toc}{section}{ABSTRACT}
\begin{center}
\zihao{3}\rmfamily\bfseries
ABSTRACT
\end{center}

\par
This is abstract. Literature are for example\cite{IOTB2}.The appendix of this article is used for example, like this "See Appendix \ref{matlab} for details".

\bigskip
\noindent\textbf{Key Words:}\quad Key1\quad Key2\quad Key3\quad Key4

\newpage

\phantomsection%设置“幻影标题”,解决额外插入目录(\addcontentsline)后总是引用到第一页的问题;
\addcontentsline{toc}{section}{目录}
\vspace*{-21.6pt}
\tableofcontents
\newpage

\setcounter{page}{1}
\pagenumbering{arabic}
\cfoot{\small 第\,\thepage\,页\quad 共\,\pageref{LastPage}\,页}
\newpage\vspace*{-21.6pt}

\section{标题1}
\subsection{小标题1}
正文部分1,
\subsection{小标题2}
\subsubsection{小小标题1}
正文部分2。
\subsubsection{小小标题2}
正文部分3,Matlab代码示例见附录\ref{python},作为附录引用示例。
\newpage
\section{标题2}
\subsection{小标题1}
中间部分,开始摆烂(doge),随便引用一点文献\cite{IOTJA}。
\subsection{小标题2}
……

内容就靠自己了,文档格式来源于“中南大学毕业论文(模板).docx”

\newpage
\begin{center}
	\zihao{3}\heiti
	\phantomsection
	\addcontentsline{toc}{section}{结束语(或致谢)}
	结束语(或致谢)
\end{center}

这里写结束语或致谢内容。这里写结束语或致谢内容。这里写结束语或致谢内容。这里写结束语或致谢内容。这里写结束语或致谢内容。这里写结束语或致谢内容。这里写结束语或致谢内容。这里写结束语或致谢内容。这里写结束语或致谢内容。这里写结束语或致谢内容。这里写结束语或致谢内容。这里写结束语或致谢内容。这里写结束语或致谢内容。这里写结束语或致谢内容。这里写结束语或致谢内容。这里写结束语或致谢内容。可以顺便致谢一下这个模板(doge)。


\newpage
\phantomsection
\addcontentsline{toc}{section}{参考文献}
\bibliography{./literature/thesis}
\nocite{*}
\appendix
\label{cod:matlab}
\newpage
\vspace*{-21.6pt}
\begin{center}
	\zihao{3}\heiti
	\phantomsection
	\addcontentsline{toc}{section}{附录A:Matlab代码示例}
	附录A:Matlab代码示例
\end{center}

\definecolor{DarkGreen}{rgb}{0.0,0.4,0.0}
\lstloadlanguages{Matlab}
\lstset{language=Matlab,
	frame=shadowbox,                           % shadowbox framed
	rulesepcolor= \color{gray},%框的颜色
	breaklines=true,
	basicstyle=\small\ttfamily,
	keywordstyle=[1]\color{blue}\bfseries,  % primitive funs in bold blue
	keywordstyle=[2]\color{purple},         % args of funs in purple
	keywordstyle=[3]\color{blue}\underbar,  % user funs in blue with underbar
	stringstyle=\color{purple},             % strings in purple
	showstringspaces=false,
	identifierstyle=,
	commentstyle=\usefont{T1}{pcr}{m}{sl}\color{DarkGreen}\small,
	tabsize=4,
	% more standard MATLAB funcs
	morekeywords={sawtooth, square},
	% args of funcs
	morekeywords=[2]{on, off, interp},
	% user funcs
	morekeywords=[3]{FindESS, homework_example},
	morecomment=[l][\color{blue}]{...},     % line continuation (...) like blue comment
	numbers=left,
	numberstyle=\tiny\color{blue},
	firstnumber=1,
	stepnumber=1,
	escapeinside=``,
}

\begin{lstlisting}[language=Matlab]
	%这个是行列式转化的代码,作为示例
	function f=det2poly(P)
	n=size(P,4);
	f=0;
	if n==2  
	f=conv(P(:,:,1,1),P(:,:,2,2))-conv(P(:,:,1,2),P(:,:,2,1));  
	else
	P1=P;
	for i=1:n    
	P=P1;     
	P(:,:,1,:)=[];
	P(:,:,:,i)=[];   
	f=(-1)^(1+i)*conv(P1(:,:,1,i),det2poly(P))+f;   
	end
	end
\end{lstlisting}

\newpage
\begin{center}
	\zihao{3}\heiti
	\phantomsection
	\addcontentsline{toc}{section}{附录B:Python代码示例}
	附录B:Python代码示例
\end{center}

\begin{python}
	import math
	
	
	def hix(x):
	if x > 0:
	return x
	else:
	return 0
	
	
	def trans1(x1, x2, n, s1, s2, lock):
	# 1.	(A1,B1) -> (A1,B1)
	if lock == 1:
	dt = 22 * (math.floor(x1 / 2) + math.floor(x2 / 2))
	dw = math.floor(x1 / 2) * n + math.floor(x2 / 2) * (150 - n)
	return dt, dw, lock
	else:
	return 0, 0, lock
\end{python}
\end{document}
